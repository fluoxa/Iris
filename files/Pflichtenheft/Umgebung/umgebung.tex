 
\section{Umgebung}
 
\subsection{Software}

Die Anwendung untersteht keinen Restriktionen, Sie kann mit jedem Betriebssystem ausgeführt werden.
Abhängig der Zielgruppe entstehen die folgenden Softwarevoraussetzungen.

\subsubsection {Für Endanwender}

\begin{itemize}
\item aktueller Internet Browser 
\item bestehende Internetverbindung
\end{itemize}

\subsubsection	{Für Entwickler}


\begin{table}[h]
\renewcommand{\arraystretch}{1.5}
\begin{tabular}{llr}
\textbf{Software} & \textbf{Version} \\ \hline
Java              & 8                \\
Apache Maven      & 3.3.9            \\
Spring Framework  & 1.4.1            \\ 
Vaadin            & 7.7.3            \\
FasterXML         & 0.6.2            \\
Lombok            & 1.16.10          \\
TestNG            & 6.9.13.6         \\
Mockito           & 2.0.2 beta       \\
dozer             & 5.5.1            \\
Apache Commons-io & 2.5             
\end{tabular}
\end{table}
 
\subsection{Hardware}

Es gelten keine besonderen Hardwarevoraussetzungen. Die Anwendung ist mit handelsüblichen PC ausführbar. Entwickler benötigen darüber hinaus einen Application-Server mit Webanbindung.

\subsection{Orgware}

\begin{itemize}
\item Web-Repository GitHub zur Versionsverwaltung
\item gitter.im zur internen Kommunikation 
\end{itemize}
