 
\section{Umgebung}
 
\subsection{Software}

Die Anwendung untersteht keinen Restriktionen, Sie kann mit jedem Betriebssystem ausgeführt werden.
Abhängig der Zielgruppe entstehen die folgenden Softwarevoraussetzungen.

\subsubsection {Für Endanwender}

\begin{itemize}
\item aktueller Internet Browser 
\item bestehende Internetverbindung
\end{itemize}

\subsubsection	{Für Entwickler}

\begin{table}[H]
\begin{center}
\begin{tabular}{|ll|}
\hline \hline \cellcolor{blue!25} & \cellcolor{blue!25}\\[-0.4cm]
\cellcolor{blue!25} \textbf{Software} &  \cellcolor{blue!25} \textbf{Version} \\ 
\hline& \\[-0.4cm]
Java              & 8                \\[0.1cm]
Apache Maven      & 3.3.9            \\[0.1cm]
Spring Framework  & 1.4.1            \\[0.1cm] 
Vaadin            & 7.7.3            \\[0.1cm]
FasterXML         & 0.6.2            \\[0.1cm]
Lombok            & 1.16.10          \\[0.1cm]
TestNG            & 6.9.13.6         \\[0.1cm]
Mockito           & 2.0.2 beta       \\[0.1cm]
dozer             & 5.5.1            \\[0.1cm]
Apache Commons-io & 2.5 \\
\hline \hline             
\end{tabular}
\end{center}
\end{table}
 
\subsection{Hardware}

Es gelten keine besonderen Hardwarevoraussetzungen. Die Anwendung ist mit handelsüblichen PC ausführbar. Entwickler benötigen darüber hinaus einen Application-Server mit Webanbindung.

\subsection{Orgware}

\begin{itemize}
\item Web-Repository GitHub zur Versionsverwaltung
\item gitter.im zur internen Kommunikation 
\end{itemize}
