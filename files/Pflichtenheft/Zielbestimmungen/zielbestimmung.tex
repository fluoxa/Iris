\section{Zielbestimmung}
Dieses Kapitel dient der Bestimmung von Zielen und nicht für deren Verwendung
notwendige Funktionen.
 
\subsection{Musskriterien}
%blubla Musskriterien: Für das Produkt unabdingbare Leistungen, die in jedem Fall
%erfüllt werden müssen \footnote{gezwungen sein, etwas zu tun (Dies ist eine
%beispielhafte Fußnote).}. Das System ist ohne diese Funktionen für seinen
%gedachten Zweck nicht einsetzbar.

Um dem Ziel, der Erkennung einer handegschriebenen Ziffer, gerecht zu werden, müssen 
folgende Kriterien erfüllt sein:

\begin{itemize}
\item Möglichkeit zur Erfassung einer handgeschriebenen Ziffer in einer Weboberfläche
\item Umwandlung der Eingabeziffer in eine maschinenlesbare Form mithilfe neuronaler Netze
\item Ausgabe der Resultate in einer Weboberfläche
\item trainieren der neuronalen Netzten
\item Persistierung von neuronalen Netzen
\end{itemize}
 
\subsection{Kannkriterien}
%Kannkriterien: Die Erfüllung der Kannkriterien ist erwünscht, jedoch nicht
%unbedingt notwendig. Sie sollten nur angestrebt werden, falls noch ausreichend
%Kapazitäten vorhanden sind.

Sind die Musskriterien erfüllt und noch freie Kapazitäten vorhanden, werden folgende Kriterien bearbeitet:

\begin{itemize}
\item Internationalisierung zur Steigerung der Usability
\item Erstellung und Bearbeitung von neuronalen Netzten
\item Bereitstellung der Möglichkeit, verschieden trainierte neuronale Netze auszuwählen
\item Suchfunktion über vorhandene Netze in der Weboberfläche
\end{itemize}

 
\subsection{Abgrenzungskriterien}
Im späteren Entwicklungszyklus der Anwendung sollen die Funktionalitäten erweitert werden. Denkbar wäre die Möglichkeit
der Eingabe einfacher mathematischer Operationen, welche durch das neuronale Netz erkannt und berechnet werden.
