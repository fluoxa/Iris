\section{Einsatz}
Der geplante Einsatz des Systems ist die Grundlage für Benutzungsoberfläche und
Qualitätsanforderungen.
 
\subsection{Anwendungsbereiche}

Das System ist über einen mit dem Internet verbundenen Rechner aufrufbar. Dadurch kann die Anwendung von allen Personen mit Internetzugang genutzt werden.

\subsection{Zielgruppen}

``Iris'' ist in erster Linie für den Endanwender gedacht. Die Anwendung soll daher für jeden Nutzer zugänglich sein, der Nutzen aus einer handschriftlichen Ziffernerkennung ziehen kann.\\
Da ``Iris'' im Kern auf den Einsatz neuronaler Netze setzt, sollen auch technisch versierte Informatiker angesprochen werden, die sich mit neuronalen Netzen beschäftigen möchten und eventuell an der Weiterentwicklung von ``Iris'' teilhaben wollen.

\subsection{Betriebsbedingungen}

Da es sich um eine Internetanwendung handelt, ähneln sich die Betriebsbedingungen mit jenen anderer Internetanwendung.

\begin{itemize}
	\item unbeaufsichtiger, wartungsfreier, 24 Stunden am Tag Betrieb\\[-0.9cm]
	\item Server mit Webanbindung \\[-0.9cm]
	\item laufende mongoDB (Version 3.2.10)
\end{itemize}
