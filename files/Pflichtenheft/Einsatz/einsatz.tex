\section{Einsatz}
Der geplante Einsatz des Systems ist die Grundlage für Benutzungsoberfläche und
Qualitätsanforderungen.
 
\subsection{Anwendungsbereiche}
%Ein Pflichtenheft wird bspw. in einer IT-Abteilung genutzt.

Das System ist über einen, mit dem Internet verbundenen Rechner aufrufbar.
Dadurch kann die Anwendung von allen Personen mit Internetzugang genutzt werden.


 
\subsection{Zielgruppen}
%Die Zielgruppe besteht also aus Informatikern, die mit der Projektplanung
%beauftragt wurden.

Die größte Zielgruppe ist der Standartnutzer ohne technische Vorkenntnisse, welcher sich für eine Erkennung handschriftlicher Ziffern interessiert.
Da es sich um ein Open-Source Projekt handelt, sollen auch technisch versierte Informatiker angesprochen und zur Teilnahme an der Entwicklung 
der Handschriftserkennung mithilfe neuronaler Netze angeregt werden.

\subsection{Betriebsbedingungen}
%Betriebsbedingungen: Die Betriebsbedingungen spezifiziert die physikalische
%Umgebung des Systems, die tägliche Betriebszeit, und ob das System ständiger
%Beobachtung durch Bediener ausgesetzt ist, oder ein unbeaufsichtigter Betrieb
%beabsichtigt ist.

Die Grundvorrausetzung zur Inbetriebnahme der Anwendung ist ein Server mit Webanbindung. Auf diesem muss mindestens Java 8 installiert sein.
Zur Persistierung der Daten wird mongoDB der Version 3.2.10 verwendet. Der Betrieb der Datenbank muss dauerhaft gewährleistet und die Daten
jederzeit vom Server abrufbar sein. Es wird ein unbeaufsichtiger Betrieb angestrebt. 
