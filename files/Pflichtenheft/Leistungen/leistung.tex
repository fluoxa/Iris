\section{Leistungen}
Anforderungen bezüglich der Zeit und Genauigkeit. \\[-0.2cm]

\textbf{/L101/} Die Webanwendung soll innerhalb 5s Sekunden geladen und einsatzbereit sein. \\[-0.2cm]

\textbf{/L102/} Die in /F102/ beschriebene Ziffernberechnung soll in maximal 30ms erfolgen.\\[-0.2cm]

\textbf{/L103/} Die Genauigkeit der Ziffernerkennung soll beim Test gegen die MNIST-Test\-datenbank\footnote{Vgl. Fußnote \ref{footnote_mnist}.} mindestens 95\% betragen.\\[-0.2cm]

\textbf{/L104/} Die Ausgabe des Ergebnisses soll wahlweise in Echtzeit erfolgen.\\[-0.2cm]

\textbf{/L105/} Die in /F105/ beschriebene Suche soll auch bei vielen Netzen ($n > 100$), innerhalb weniger Sekunden ($t<4$s), ein Ergebnis liefern.\\[-0.2cm]

\textbf{/L106/} Die in /F106/ beschriebenen Statusinformationen sollen in Echtzeit ausgegeben werden.\\[-0.2cm]

\textbf{/L201/} Das in /F201/ beschriebene Training soll bei maximaler Trainingsbilderanzahl (60000 Stck.) und minimaler Minibadgegröße (1 Stck.) in maximal 10s abgeschlossen sein.\\[-0.2cm]

\textbf{/L301/} Jede implementierte Funktion soll durch Test auf richtige Funktionsweise überprüfbar sein. Eine Funktionsabdeckung von nahezu 95\% und eine Zweigabdeckung von 70\% wird angestrebt.\\[-0.2cm]
